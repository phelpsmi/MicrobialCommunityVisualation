%ten point, letter paper, single column
\documentclass[letterpaper,10pt, onecolumn, draftclsnofoot]{IEEEtran}

\usepackage{graphicx}                                        
\usepackage{amssymb}                                         
\usepackage{amsmath}                                         
\usepackage{amsthm}                                          

\usepackage{alltt}                                           
\usepackage{float}
\usepackage{color}
\usepackage{fancyvrb}
\usepackage{url}

\usepackage{balance}
\usepackage[TABBOTCAP, tight]{subfigure}
\usepackage{enumitem}
\usepackage{pstricks, pst-node}
\usepackage{listings}
\usepackage{tabularx}

\usepackage{geometry}
%0.75 in margin on each side, total of 1.5 in
\geometry{textheight=8.5in, textwidth=6in}
\usepackage{pgfgantt}

%random comment

\newcommand{\cred}[1]{{\color{red}#1}}
\newcommand{\cblue}[1]{{\color{blue}#1}}

\newcommand{\toc}{\tableofcontents}

%\usepackage{hyperref}

\def\name{Thomas Albertine}
\title{The Many Faces of Microbial Communities \\\large Senior Design\\Spring Term Final Report\\}
\author{\name, Michael Phelps}


%% The following metadata will show up in the PDF properties
% \hypersetup{
%   colorlinks = false,
%   urlcolor = black,
%   pdfauthor = {\name},
%   pdfkeywords = {capstone visualization microbiology microbio senior design},
%   pdftitle = {Senior Design Project, Visualizing Microbial Communities},
%   pdfsubject = {Senior Design Microbiology Visualizations},
%   pdfpagemode = UseNone
% }

\lstset{language=python,
                basicstyle=\ttfamily,
                keywordstyle=\color{blue}\ttfamily,
                stringstyle=\color{red}\ttfamily,
                commentstyle=\color{green}\ttfamily,
                morecomment=[l][\color{magenta}]{\#},
				breaklines=true,
				showstringspaces=false
}

\parindent = 0.0 in
\parskip = 0.1 in

%Single space
\linespread{1.0}

\input{pygments.tex}

\begin{document}
\maketitle
\section{Abstract}

\clearpage

\tableofcontents

\section{Introduction}
Our client, Dr. Jenna Lang, a microbiology researcher from University of California Davis requested that we create a tool that she could use to visualize microbial population data as human faces. This tool could potentially help microbiology researchers (as well as those from other fields) find patterns that aren't obvious using existing visualization techniques. 

Currently, the standard visualizations are pie charts or heat maps. Unfortunately, Pie charts aren't effective when there are a large number of categories to represent because there are too many slices. Heat maps on the other hand, while still useful to see patterns, do not make proportions between organisms obvious the way pie charts do. Our tool takes advantage of the human brain's ability to recognize and interpret patterns in faces to visualize data, so users can find new patterns that they would normally miss.

Our team consisted of Thomas Albertine and Michael Phelps. We had no formal roles aside from ``developer'', although generally (but not exclusively) Thomas worked on the backend code, loading data files and generating models, while Michael worked on the frontend code, connecting the backend to a user interface (henceforth referred to as UI). Our client played a hands off role. We spoke with her early in the project to collect requirements and have contacted her to revisit requirements or to keep her aprised of how the project is going, but she generally gave usa lot of freedom regarding how to proceed.

\section{Original Requirements Document}
\subsection{Gantt Chart}
\begin{figure}[h]
	\begin{ganttchart}[vgrid={*1{black, dotted}}]{1}{21}
		\gantttitle{Initial Gantt Chart in Weeks}{21}\\
		\gantttitlelist{1,...,21}{1}\\
		
		%\ganttbar{Task Name}{start}{end}[name] \\
		%\ganttlink{name1}{name2 defaults to elem1 or the like}
		\ganttbar[name=fformats]{Explore File Formats}{1}{2} \\
		\ganttbar[name=exploremhuman]{Explore MakeHuman API}{1}{4} \\
		\ganttbar[name=lfile]{Load File}{3}{3}\\
		\ganttbar[name=mhumanwrapper]{MakeHuman Wrapper}{5}{8}\\
		\ganttbar[name=ui]{Basic UI}{9}{10}\\
		\ganttbar[name=ldmodel]{Load and Draw Model}{11}{14}\\
		\ganttbar[name=rpsmodel]{Rotate / Pan / Scale Model}{15}{16}\\
		\ganttbar[name=polish]{Polish UI}{17}{21}\\
		\ganttbar[name=expfsprt]{Expand File Type Support}{17}{21}\\
		
		\ganttlink{fformats}{lfile}
		\ganttlink{lfile}{mhumanwrapper}
		\ganttlink{exploremhuman}{mhumanwrapper}
		\ganttlink{mhumanwrapper}{ldmodel}
		\ganttlink{ui}{ldmodel}
		\ganttlink{ui}{polish}
		\ganttlink{ldmodel}{rpsmodel}
		\ganttlink{rpsmodel}{polish}
	\end{ganttchart}
	\caption{This is the original Gantt chart for the project.}
\end{figure}

\subsection{Document}

\section{Final Requirements Document}

\section{Design Document}

\section{Tech Review}

\section{Blog Posts}

\section{Poster}

\section{Project Documentation}

\section{Learning New Technologies}

\section{Personal Learning}
\subsection{Thomas Albertine}

\subsection{Michael Phelps}

\appendix[Essential Code Listings]

\appendix[Additional Content]

\end{document}
